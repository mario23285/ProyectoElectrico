% EL RESUMEN EN INGLÉS
% --------------------

\begin{theabstract}{Design And Implementation Of A Motion Capture Glitches Correction Algorithm}{Digital Animation, Motion Capture, 3D Interpolation, Computer Graphics}

Nowdays, motion capture technology, is used in productions of all levels in Digital Animation. The concept on which this technology is based on, consists of the elaboration of 3D models from numerical data interpreted by a set of infrared cameras and sensors. The problem with this technology is that the processing of the data generated by these sensors is not always correct, which causes loss of position data which results in corrupted 3D models.

This project consists of the design and implementation of a software program capable of correcting such camera processing errors, in order to generate accurate 3D models, by using 3D interpolation algorithms.

The theoretical resources to propose a solution to this problem, are based on the interpolation algorithms known as Basic Spline or B-Spline. For the software implementation, it was decided to use C / C++, as it presents several performance advantages, better Computer Graphics libraries and compatibility with other research tools, currently being developed in the Pattern Recognition and Intelligent Systems Laboratory ( PRIS-Lab).

The expected result is a program capable of receiving a motion capture file as the input argument, and outputting a corrected BioVision Hierarchical Data file.


\end{theabstract}

% El entorno 'theabstract' tiene el formato \begin{theabstract}{A} ...B... \end{theabstract} donde A es el título del proyecto traducido de inglés a español y B es el contenido, en inglés, del resumen. Se recomienda buscar ayuda calificada para la elaboración y/o revisión de este resumen.