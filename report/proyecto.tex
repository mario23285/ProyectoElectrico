% ------------------------------------
% UNIVERSIDAD DE COSTA RICA
% Facultad de Ingeniería
% Escuela de Ingeniería Eléctrica
% IE0499 - Proyecto Eléctrico
%
% PLANTILLA Y GUÍA DEL TRABAJO ESCRITO
% Versión: v0.8 (marzo 2017)
% ------------------------------------

% Tipo de documento
\documentclass[final]{proyectoelectrico}

% A. PAQUETES Y MACROS ESPECIALES -------
% Paquetes y definiciones que no están incluidos 
% en la clase proyectoelectrico.cls o que son 
% propios del proyecto.
\input{configuracion.tex}

% Tipografía
\usepackage{libertine}
\usepackage{libertinust1math}
\usepackage[T1]{fontenc}
\renewcommand*{\ttdefault}{cmtt}
% ---------------------------------------

% B. DATOS ------------------------------
% Todos los nombres incluyen dos apellidos,
% acentos y signos de puntuación apropiados.
% Revisar las recomendaciones sobre el título
% y los nombres de los profesores en la guía.

% Título del proyecto
\titulo{Diseño e Implementación de un algoritmo de corrección de errores de captura de movimiento}

% Autor (nombre y carné)
\autor{Mario Alberto Castresana Avendaño}
\carne{A41267}

% Profesor(a) guía
% (Ing.) Nombre Apellido Apellido(, título)
\guia{Francisco Siles Canales, Dr.rer.nat}

% Profesores lectores 
\lectorA{Ing. José Fernando Salas Fumero}
\lectorB{Ricardo Román Brenes, M. Sc.}

% Fecha de entrega del trabajo escrito
\mes{7}		% Número del mes
\ano{2017}	% Formato AAAA
% ---------------------------------------

% C. CONTENIDOS -------------------------

%%%%%%%%%%%%%%%%%%%
\begin{document}
%%%%%%%%%%%%%%%%%%%

\frontmatter

% 1. PORTADA
\portada

% 2. HOJA DE APROBACIÓN
\iffinal
\aprobacion
\fi

% 3. RESUMEN (EN ESPAÑOL E INGLÉS)
% EL RESUMEN
% ----------

\begin{resumen}{Animación Digital, \textit{Motion Capture}, Interpolación 3D, \textit{Computer Graphics}}

En la actualidad, la tecnología de captura de movimiento o \textit{Motion Capture}, se utiliza en producciones de todo nivel en Animación Digital.  El concepto en el que se basa dicha tecnología, consiste en la elaboración de modelos 3D a partir de datos numéricos interpretados por un conjunto de cámaras infrarrojas y sensores.  El problema que presenta esta tecnología es, que el procesamiento de los datos arrojados por estos sensores no siempre es correcto, lo cual hace que existan pérdidas de datos de posición que resultan en modelos 3D defectuosos.

Este proyecto, consiste en el diseño e implementación de un programa de \textit{software} capaz de corregir dichos errores de procesamiento por parte de las cámaras, y con esto, generar modelos de animación 3D corregidos mediante el uso de algoritmos de interpolación en 3D.

Los recursos teóricos para proponer una solución a este problema, se basan en los algoritmos de interpolación conocidos como \textit{Basic Spline} o \textit{B-Spline}.  Para la implementación del programa como tal, se decidió utilizar C/C++, ya que presenta varias ventajas en desempeño, bibliotecas de \textit{Computer Graphics} y compatibilidad con otras herramientas de investigación, actualmente desarrolladas en el laboratorio de investigación en reconocimiento de patrones y sistemas inteligentes (\textit{PRIS-Lab}).

El resultado que se espera obtener, es un programa capaz de recibir un archivo de captura de movimiento como argumento de entrada, y producir a la salida, un archivo corregido tipo \textit{\textbf{B}io\textbf{v}ision \textbf{H}ierarchical Data}.

\end{resumen}
% EL RESUMEN EN INGLÉS
% --------------------

\begin{theabstract}{Design And Implementation Of A Motion Capture Glitches Correction Algorithm}{Digital Animation, Motion Capture, 3D Interpolation, Computer Graphics}

Nowdays, motion capture technology, is used in productions of all levels in Digital Animation. The concept on which this technology is based on, consists of the elaboration of 3D models from numerical data interpreted by a set of infrared cameras and sensors. The problem with this technology is that the processing of the data generated by these sensors is not always correct, which causes loss of position data which results in corrupted 3D models.

This project consists of the design and implementation of a software program capable of correcting such camera processing errors, in order to generate accurate 3D models, by using 3D interpolation algorithms.

The theoretical resources to propose a solution to this problem, are based on the interpolation algorithms known as Basic Spline or B-Spline. For the software implementation, it was decided to use C / C++, as it presents several performance advantages, better Computer Graphics libraries and compatibility with other research tools, currently being developed in the Pattern Recognition and Intelligent Systems Laboratory ( PRIS-Lab).

The expected result is a program capable of receiving a motion capture file as the input argument, and outputting a corrected BioVision Hierarchical Data file.


\end{theabstract}

% El entorno 'theabstract' tiene el formato \begin{theabstract}{A} ...B... \end{theabstract} donde A es el título del proyecto traducido de inglés a español y B es el contenido, en inglés, del resumen. Se recomienda buscar ayuda calificada para la elaboración y/o revisión de este resumen.

% 4. RECONOCIMIENTOS
\iffinal
% LOS RECONOCIMIENTOS
% -------------------

% Aquí se escribe la dedicatoria del proyecto y los agradecimientos. El entorno 'reconocimiento' tiene la estructura \begin{reconocimiento}{Dedicatoria} Agradecimientos \end{reconocimiento}

\begin{reconocimiento}{Dedicado a mi familia y amigos.}

\lipsum[5]

\end{reconocimiento}
\fi

% 5. TABLAS DE CONTENIDO, FIGURAS Y TABLAS
\tableofcontents
\listoffigures
\listoftables

% 6. NOMENCLATURA
% LA NOMENCLATURA
% ---------------

% La nomenclatura se realiza con el paquete 'nomencl'. Para ingresar un nuevo elemento, se debe usar el comando \nomenclature{símbolo}{definición}, ya sea en este archivo nomenclatura.tex (más fácil para encontrar y editar), o en cualquier parte del documento (probablemente cuando se introduce una nueva variable o constante). Para más opciones del paquete, favor referirse a su documentación (https://www.ctan.org/pkg/nomencl). También hay una buena guía de uso en https://www.sharelatex.com/learn/Nomenclatures.

% Formato recomendado
% -------------------

% Variable o constante matemática
% \nomenclature{$V$}{Tensión eléctrica}

% Acrónimo
% \nomenclature{TBH}{Para ser honesto (del inglés \textit{To Be Honest})}

% Si únicamente existen acrónimos del inglés, se puede omitir la frase 'del inglés'. La definición no tiene punto al final.

\nomenclature{$c$}{Velocidad de la luz}
\nomenclature{$h$}{Constante de Planck}
\nomenclature{$R$}{Resistencia eléctrica}
\nomenclature{$I$}{Corriente eléctrica}
\nomenclature{$V$}{Tensión eléctrica}
\nomenclature{$v_s(t)$}{Función sinusoidal}
\nomenclature{$T_0$}{Temperatura ambiente}
\nomenclature{$V$}{Tensión eléctrica}
\nomenclature{$e$}{Carga elemental}
\nomenclature{$\epsilon_0$}{Constante eléctrica (permitividad)}
\nomenclature{$\mu_0$}{Constante magnética (permeabilidad)}
\nomenclature{$G$}{Constante gravitacional newtoniana}
\nomenclature{$\mathcal{N}$}{Distribución gaussiana}
\nomenclature{$V$}{Tensión eléctrica}
\nomenclature{IEEE}{Instituto de Ingenieros Eléctricos y Electrónicos (del inglés \textit{Institute of Electrical and Electronics Engineers})}
\nomenclature{EIE}{Escuela de Ingeniería Eléctrica de la Universidad de Costa Rica}
\nomenclature{LOS}{Línea de vista (del inglés \textit{Line Of Sight})}
\nomenclature{SVP}{Por favor (del francés \textit{S'il Vous Pla\^{i}t})}


\printnomenclature

\mainmatter

% 7. CAPÍTULOS
% ----------------------
  \chapter{Introducción}
% ----------------------
\label{C:introduccion}

Actualmente, en el laboratorio de investigación en  reconocimiento de patrones y sistemas inteligentes (\textit{PRIS-Lab}), se utilizan cámaras infrarrojas de captura de movimiento para el estudio del movimiento humano. Dichas cámaras son capaces de generar un modelo 3D básico del esqueleto humano, que se reprensenta en archivos del tipo  \textit{\textbf{B}io\textbf{v}ision \textbf{H}ierarchical Data} o \textbf{BVH}.  Los archivos BVH contienen los datos necesarios para generar un modelo 3D que captura fielmente todos los movimientos hechos por un actor enfrente de las cámaras y se usan frecuentemente en Animación Digital y otras áreas afines.

El problema con estás cámaras es que, por lo general, las grabaciones presentan una cantidad considerable de errores de captura o \textit{glitches}, en los cuales la posición de los sensores infrarrojos no es capturada de manera correcta y eso deriva en modelos en los que las articulaciones se deforman o hacen movimientos antinaturales.  El presente proyecto, trata sobre el diseño e implementación de un algoritmo capaz de eliminar esos errores de captura de movimiento de un archivo BVH, y así generar una animación 3D más correcta y fluida.  

En general, este proyecto se mantiene dentro del área de la computación y otras ramas de investigación afines al \textit{PRIS-Lab}; mediante el uso de técnicas de desarrollo de \textit{software} y algoritmos del área de \textit{Computer Graphics}, se pretende crear un programa en C/C+ + capaz de resolver el problema propuesto. 



%%%%%%%%%%%%%%%%%%%%%%%%%%%%%%%%%%%%%%%%%%%%%%
\section{Alcances del proyecto}
%%%%%%%%%%%%%%%%%%%%%%%%%%%%%%%%%%%%%%%%%%%%%%

Los archivos de tipo \textbf{BVH}, por lo general presentan varios tipos de errores de captura de movimiento.  Como parte de los objetivos de este proyecto, se pretende definir una taxonomía de errores adecuada, con el fin de delimitar puntualmente qué tipo de eventos se consideran errores y cuáles de éstos se van a corregir. De forma general, el presente proyecto se limita a la corrección de errores espaciales y temporales conocidos como \textit{glitches}.

Cabe mencionar, que no será parte de este trabajo diseñar o implementar una metodología o algoritmo para la detección de errores en un archivo \textbf{BVH}.  El cómo detectar los errores y delimitarlos temporalmente, quedará para otro proyecto dada su complejidad.  Tampoco se pretende crear un programa que sustituya en su totalidad las capacidades de un animador profesional.

\subsection{Herramientas a utilizar}

La manipulación de datos puros para generar un modelo en 3D, requiere de un conjunto de bibliotecas estandarizadas de \textbf{OpenGL} para funcionar de manera eficiente.  La disponibilidad de dichos recursos, depende del lenguaje de programación que se escoja para la implementación del programa, sin embargo, debido a las herramientas que actualmente se desarrollan en el \textit{PRIS-Lab}, se escogió desarrollar el programa en C/C++.

A parte de las necesidades del laboratorio, se escogió C/C++ debido a que presenta varias bibliotecas muy versátiles en \textbf{OpenGL} y otras herramientas gráficas comunes, sin mencionar el hecho de que muchas de las bibliotecas que existen en C/C++ son de Código Abierto.

Partiendo de esta selección, inicialmente se escogieron como punto de partida las bibliotecas GLEW (\textbf{OpenGL}), GLM y Einspline para trabajar en la solución propuesta.  Las dos primeras bibliotecas, son herramientas comunes en el área de \textit{Computer Graphics}, mientras que Einspline, se centra en resolver todo tipo de problemas de interpolación en 1D, 2D y 3D, usando \textit{B-Splines}.

Dentro de los aspectos a omitir en este proyecto, se encuentran el diseño de una interfaz gráfica para el programa, ya que se supone que éste correrá en el \textit{cluster} del \textit{PRIS-Lab}.  Tampoco se implementarán varios tipos de interpolación para la manipulación de curvas de animación.

%%%%%%%%%%%%%%%%%%%%%%%%%%%%%%%%%%%%%%%
\section{Justificación}
%%%%%%%%%%%%%%%%%%%%%%%%%%%%%%%%%%%%%%%

La tecnología de captura de movimiento o \textit{Motion Capture}(\textit{MoCap}), es muy utilizada en áreas como la Animación Digital para diferentes tipos de producciones audiovisuales, que van desde los videojuegos, comerciales de televisión, hasta el cine.  Normalmente, muchos animadores se basan en modelos 3D obtenidos con esta tecnología para producir todo tipo de animaciones.

El problema que este proyecto pretende resolver, es importante porque en el área de Animación Digital, corregir los archivos de \textit{MoCap} representa una inversión significativa en cualquier producción.  Poner a un animador profesional a corregir manualmente, cuadro por cuadro, un archivo \textbf{BVH} puede costar \$10 por segundo, por personaje animado.  

Esto quiere decir, que una producción pequeña como un comercial o un corto animado, puede llegar a costar miles de dólares, solo por el hecho de solicitar a un animador depurar un archivo de \textit{MoCap}. Eso sin mencionar, que es un proceso muy tedioso.

Revisando las herramientas comerciales disponibles y la literatura existente, no se pudo encontrar una herramienta de \textit{software} que eliminara errores de captura de un archivo de \textit{Motion Capture}.  Obviamente, no se han podido hacer programas que sustituyan a los animadores.  Lo que sí se puede hacer, es un programa que elimine los errores más comunes de una captura de movimiento y produzca un archivo \textbf{BVH} de mejor calidad.

En este sentido, el aporte principal del presente proyecto, se basa en crear dicho programa, y con eso, tratar de aligerar la carga que supone para un grupo de producción o investigación, lidiar con errores de captura de movimiento.



\section{Objetivos}

\subsection{Objetivos generales}

% Objetivos generales----------<<<
\begin{itemize}
\item Realizar una investigación bibliográfica referente a los algoritmos más comunes para la interpolación de curvas de animación.
\item Seleccionar los algoritmos a implementar para la interpolación de curvas de animación.
\item Implementar de dichos algoritmos en C/C++.
\item Validar la implementación en pruebas de ejecución.
\item Escribir un artículo para la rama estudiantil de la IEEE  (CONESCAPAN).
\end{itemize}
% --------------------<<<

\subsection{Objetivos específicos}

%Objetivos específicos---------<<<
\begin{itemize}
\item Definir una taxonomía de \textit{glitches} para delimitar qué tipo de errores el programa será capaz de corregir.
\item Documentar toda la información necesaria para la implementación del programa en C/C++.

\end{itemize}
%---------------------<<<

\section{Metodología}

El diseño del programa, se trabajará con metodologías de desarrollo ágiles (\textit{Agile}), en las cuales, se definirán un conjunto de tareas, con sus respectivos avances por semana. Los avances se darán a conocer semanalmente en las reuniones de PRIS-ProSeminar, donde se dará retroalimentación de cada tarea desarrollada, así como la revisión de objetivos y del trabajo escrito.
% ----------------------
  \chapter{Teoría del trabajo escrito} 
% ----------------------
\label{C:antecedentes}

Uno o más capítulos del trabajo escrito deben dedicarse a la teoría y desarrollos que sustentan el proyecto. Una de las preguntas más usuales es ``¿qué hay que incluir en la teoría?'' Esta es una decisión en primera instancia del estudiante junto con el profesor guía. A continuación hay otros criterios para tomar estas decisiones.

\section{¿A quién va dirigido el trabajo escrito?}

Una guía útil para tomar decisiones sobre qué incluir en el trabajo escrito es tener conciencia de cuál es el público meta de la lectura. Esto marca grandes diferencias, pues, por ejemplo, para un sector externo de la carrera habría que iniciar explicando conceptos básicos; mientras que, si está dirigido a profesores, se omite casi todo tema introductorio y se enfoca en los aspectos novedosos del trabajo. Sin embargo, este último enfoque puede dejar a muchos lectores con vacíos importantes de teoría.

Tomando en consideración lo anterior, la recomendación de la Escuela es la siguiente:

\begin{quote}
\bfseries\centering
Los lectores del trabajo escrito del Proyecto Eléctrico son \\ estudiantes de ingeniería eléctrica del último año de carrera.
\end{quote}

Esta recomendación permite tomar algunas decisiones importantes, por ejemplo:

\begin{description}
\item[¿Se debe explicar el concepto de resistencia eléctrica?] No, porque se asume a todo lector familiar con el tema\footnote{A pesar de eso, la explicación de la resistencia eléctrica se ha incluido muchas veces en trabajos escritos.}. Sería justificable cuando el proyecto estudia y modifica algún concepto fundamental relacionado con la resistencia o la capacitancia (por ejemplo, en teoría de sistemas micro electromecánicos, MEMS).
\item[¿Se debe profundizar en la historia de temas como computadoras o la teoría de control?] No, si se considera que hay cursos de la carrera dedicados a estos temas, y que su explicación probablemente no hace una diferencia en el trabajo del proyecto\footnote{Por cortesía con el lector, algunos desarrollos históricos se pueden esbozar, sobre todo si dan contexto a novedades más recientes.}.
\end{description}

\section{Convenciones básicas de formato}

Por tratarse de una presentación con base en la recopilación, el análisis y la síntesis de trabajos de otros autores, la referencia adecuada a los mismos es indispensable.  Toda copia (\emph{¡plagio!}), es inaceptable.

El contenido del capítulo debe ser relevante para el proyecto y no ``material de relleno'', o incluido con el único propósito de ``engordar'' el informe.

El estado de la técnica establece el \emph{punto de partida} del estudio realizado y posiblemente también, la \emph{base de comparación} para las pruebas realizadas.

Este capítulo es importante porque muestra la capacidad de análisis y síntesis del estudiante.
 
\section{Ecuaciones}

Las ecuaciones estarán centradas y numeradas en forma secuencial por capítulo, al margen derecho. La referencia a ellas se hará utilizando su número.

\paragraph{Ejemplo} ``El modelo utilizado para representar al proceso, es de primer orden más tiempo muerto, dado por la función de transferencia de la ecuación (\ref{E:FT})

\begin{equation}
	P(s) = \frac{K \me^{-Ls}}{Ts+1}, \label{E:FT}
\end{equation}

\noindent donde $K$ es la ganancia, $T$ la ...''  

Las ecuaciones forman parte del texto, por lo que deben terminarse con el signo de puntuación requerido, una coma o un punto.

\paragraph{Ejemplos de ecuaciones}

\begin{equation}
	\tau \frac{\md T_{tc}(t)}{\md t} + T_{tc}(t) = T_{gas}(t)
\end{equation}

\begin{align}
	L_1 \frac{\md i_{L_1} (t)}{\md t} &= v(t) - R_1 i_{L_1}(t) - v_c(t), \\
	C \frac{\md v_c (t)}{\md t} &= i_L(t)- \frac{1}{R_2} v_c(t).
\end{align}

\begin{equation}
x_{1,2} = \frac{-b \pm \sqrt{b^2 - 4ac}}{2a}
\end{equation}

\begin{equation}
Z = \begin{cases}
\sqrt{\epsilon_r - \cos^2 \theta}/\epsilon_r & \text{para polarización vertical} \\
\sqrt{\epsilon_r - \cos^2 \theta} & \text{para polarización horizontal} \\
\end{cases}
\end{equation}


\section{Figuras y tablas}

Las figuras y las tablas son \emph{elementos flotantes}. Para figuras grandes se prefiera ubicarlos al inicio de la página.

\subsection{Figuras}
Las referencias a las figuras debe hacerse utilizando el número asignado a ellas.  Para esto se le asigna una etiqueta (con \texttt{label}) y luego se utiliza esta para hacer la referencia (con \texttt{ref}).  Usar en el texto el término ``figura'' y no Fig.'' o ``fig.''.

La leyenda (con \texttt{caption}) de la figura, irá en la parte inferior de la misma.  Como en forma predeterminada en la clase \texttt{eieproyecto} las figuras están centradas, no es necesario usar \texttt{centering} para hacerlo.

Por ejemplo ``Considérese el diagrama de bloques mostrado en la figura en donde el proceso controlado está dado por ...''.

No utilizar ``... en la siguiente figura ...'', emplear siempre el número correspondiente para referirse a ellas.

Cuando las figuras son muy pequeñas, se puede colocar la leyenda al lado de la misma, con el ambiente \texttt{SCfigure} del paquete \texttt{sidecap}.  Un ejemplo de esto se muestra en la figura.

Cuando un gráfico muestre varias curvas, estas deben poderse distinguir, no solamente en la pantalla de la computadora, usando diferentes colores, si no también en una impresión en blanco y negro, utilizando lineas de trazos diferentes, como se muestra en la figura.

\LaTeX~ nunca coloca las figuras y los cuadros en una página anterior a la en que son incluidas.  Los elementos flotantes los coloca en la página donde se hace referencia a ellos, o en una de las siguientes.

Además, en el texto debe hacerse referencia a todas las figuras y cuadros incluidos en el informe.  Si alguno de ellos no se menciona en el texto, es que no se requiere para entender el desarrollo presentado y por lo tanto es innecesario y se podría omitir sin que se afecte el informe.

\subsection{Tablas}
Las tablas son el otro elemento flotante utilizado en los informes y también es conveniente dejar que \LaTeX~ los coloque en donde considere que es más adecuado.

Las tablas (o cuadros) no llevarán ninguna línea divisoria vertical, solo horizontales. Una en la parte superior (\texttt{toprule}), una bajo la línea de cabecera (\texttt{midrule}) y una en la parte inferior (\texttt{bottomrule}).  Normalmente basta con estas tres líneas, pero si fuera necesaria alguna otra para una división horizontal, esta debe ser del tipo \texttt{midrule}.

Se recomienda revisar los comandos para la construcción de cuadros, incluidos en el manual de la clase \texttt{memoir} \cite{memoir2011}, o en la del paquete \texttt{booktabs} \cite{fear2005}.

La leyenda (\texttt{caption}) del cuadro se mostrará en la parte superior.  Para poder referirse al cuadro (con \texttt{ref}), se le asigna una etiqueta (con \texttt{label}).

En forma predefinida, los cuadros se mostrarán centrados horizontalmente, por lo que no es necesario hacer esa indicación. 

El cuadro \ref{tab:01} es un ejemplo de un cuadro de datos simple.

%inclusión de un cuadro con datos
\begin{table}
\caption{Parámetros de los modelos.} \label{tab:01o}
		\begin{tabular}{@{}*{4}{c}@{}}
    \toprule
    $K_p$ & $T_1$ & $T_2$ & $L$ \\
    \midrule
     1,01 & 1,50 & 0,75 & 0,12 \\
		 1,15 & 2,37 & 0,15 & 0,28 \\
		 2,25 & 5,89 & 2,15 & 1,60 \\
    \bottomrule
    \end{tabular}
\end{table}

Si la primera columna corresponde a leyendas o parámetros que identifican los datos de la línea, esta debe estar justificada a la izquierda, como se muestra en el cuadro \ref{tab:AH}, que ha sido tomada de \cite{astromhagglund2006}.

\begin{table}
\caption{Parámetros de los controladores ...} \label{tab:AH}
\begin{center}
    \begin{tabular}{@{}l*{7}{c}@{}}
    \toprule
    Controller & $K$ & $K_i$ & $K_d$ & $\beta$ & $T_i$ & $T_d$ & IAE \\
    \midrule
    PD &  1,333 & 0 & 1,333 & 1 & 0 &1 & $\infty$ \\
		PI & 0,433 & 0,192 & 0 & 0,14 & 2,25 & 0 & 6,20 \\
		PID MIGO & 1,305 & 0,758 & 1,705 & 0 & 1,72 & 1,31 & 2,25 \\
		PID $T_i=4 \ T_d$ & 1,132 & 0,356 & 0,900 & 0,9 & 3,18 & 0,80 & 2,51 \\
    \bottomrule
    \end{tabular}
\end{center}
\end{table}

Se puede especificar una cabecera para más de una columna y utilizar lineas horizontales que abarquen solo unas pocas columnas, como se muestra en el cuadro \ref{tab:muestra}.

\begin{table}
\caption{Ejemplo de otro cuadro.} \label{tab:muestra}
	\begin{tabular}{@{}l*{4}{c}@{}}
	\toprule
	& \multicolumn{2}{c}{Prueba 1} & \multicolumn{2}{c}{Prueba 2} \\
	\cmidrule(l{2pt}r{2pt}){2-3}\cmidrule(l{2pt}r{2pt}){4-5} 
	& $\Delta E=5$ V & $\Delta E = -5$ V & $\Delta E = 10$ V & $\Delta E = -10$ V \\
	\midrule
	Ganancia             &  1,06 & 0,98 & 1,12 & 0,97 \\
	Tiempo subida, s  &  5,67 & 5,89 & 6,02 & 5,74 \\
	Sobrepaso máx, \%        &  2,67 & 3,25 & 2,91 & 1,56 \\
	Error, \% &  0,25 & 0,56 & 0,97 & 0,18 \\
	\bottomrule
	\end{tabular}
\end{table}

\newpage
Cuando los cuadros son pequeños (abarcan menos de la mitad del ancho del texto), se puede colocar la leyenda a la par del cuadro, utilizando el ambiente \texttt{SCtable} del paquete \texttt{sidecap}, tal como se muestra en el cuadro \ref{tab:01}.  Compare este, con el cuadro \ref{tab:01o}.

\begin{SCtable}
\caption[Parámetros de los modelos]{Parámetros de los modelos, obtenidos a partir de las tres curvas de reacción.} \label{tab:01}
    \begin{tabular}{@{}*{4}{c}@{}}
    \toprule
    $K_p$ & $T_1$ & $T_2$ & $L$ \\
    \midrule
     1,01 & 1,50 & 0,75 & 0,12 \\
		 1,15 & 2,37 & 0,15 & 0,28 \\
		 2,25 & 5,89 & 2,15 & 1,60 \\
    \bottomrule
    \end{tabular}
\end{SCtable}
% --------------------
  \chapter{Consejos para la escritura y la investigación bibliográfica}
% --------------------
\label{C:desarrollo}

%%%%%%%%%%%%%%%%%%%%%%%%%%%%%%%
\section{Estilo del informe}
%%%%%%%%%%%%%%%%%%%%%%%%%%%%%%%

Todo el informe del proyecto debe escribirse en \emph{pasado impersonal}, siguiendo las instrucciones generales dadas en este documento. 

Debe cuidarse la redacción del informe, no solo respecto a la ortografía, sino también en cuanto a la estructura gramatical y la puntuación, la cual debe ser conforme a las reglas gramaticales del español.

\paragraph{Unidades}

Debe hacerse uso de las unidades del Sistema Internacional (SI) \cite{RTCR443-2010}, recordando que debe emplearse la coma (,), como separador decimal.

El informe puede escribirse utilizando \LaTeX~ y sus \emph{aspectos de forma} (``el formato''), están predefinidos por la clase \texttt{proyectoelectrico.cls} y no deben modificarse.

Información general sobre el uso de \LaTeX~ se encuentra en el capítulo 4 de este documento de ejemplo. Además, en el folleto de \cite{nsces}, en forma más detallada en el libro de \cite{latexcomp} y en \emph{CervanTeX - Grupo de Usuarios de \TeX~ Hispanohablantes} (\url{http://www.cervantex.es/}), entre muchas otras referencias.

El capítulo \ref{C:desarrollo} y los subsiguientes (si fueran necesarios), mostrarán el trabajo realizado en el proyecto, por lo que su cantidad, títulos y divisiones, se dejan a discreción del estudiante, con la aprobación del profesor guía y demás miembros de tribunal evaluador.

Estos capítulos muestran el ``producto'' del trabajo realizado en el proyecto, por lo cual constituyen la parte medular del informe. Debe explicarse en forma clara: qué se hizo, cómo se hizo y qué se obtuvo.

Cuando se agregan o remueven del texto elementos que aparecen en los índices (general, de figuras, de cuadros) que están en el preámbulo del informe, así como cuando se agregan o remueven citas a las fuentes bibliográficas, que aparecen en la Bibliografía al final del documento, es necesario ejecutar dos o tres veces la compilación del documento, para que estas listas se confeccionen nuevamente y se muestren correctamente.  También, en el caso de agregar o quitar ecuaciones, es necesario recompilar dos o tres veces el documento, para que se renumeren las ecuaciones y  las referencias a estas.

\section{Manejo de las fuentes bibliográficas}
Cuando se realiza un trabajo de desarrollo o investigación, siempre se parte del trabajo realizado por otras personas. Es por lo tanto indispensable, hacer referencia a las fuentes bibliográficas (referencias) utilizadas.

En \LaTeX se utiliza BibTeX para el manejo de la bibliografía.  La información de las fuentes consultadas (libros, artículos de revista o ponencias en congresos, tesis, etc.), se almacenan en un archivo \texttt{.bib} (base de datos de las fuentes bibiográficas), sin preocuparse del formato en que estas serán mostradas en el informe.  Para la creación y manejo de este archivo, se puede utilizar el programa JabRef\footnote{http://jabref.sourceforge.net/} o uno similar.

La forma en que las fuentes son listadas en el apartado Bibliografía, y como son mostradas en el texto cuando se citan, depende del \emph{estilo} seleccionado para esto.

Para el informe del proyecto eléctrico, se debe utilizar el formato APA\footnote{American Psychological Association, http://www.apa.org/}.  En inglés, este se establece utilizando el estilo \texttt{apalike}.  

\begin{equation}
 e^{jx} = \cos{x} + j \sin{x}
\end{equation}

\nomenclature{$j$}{Número imaginario.}

Junto con la clase \texttt{eieproyecto} se suministra el archivo de estilo de bibliografía \texttt{apalike\_es.bst}, en el cual se han cambiado los términos en ingles (ej. ``and'', ``In'', Edition y otros) por su equivalente en español.  Este archivo debe colocarse en la misma carpeta, en donde están los demás archivos utilizados para la confección del informe.

\begin{equation}
\int_0^\infty e^{-x^2} \mathrm{d}x = \frac{\sqrt{\pi}}{2}
\end{equation}

\begin{equation}
\mathbb{A-Z} \quad 
\mathcal{A-Z}
\end{equation}

Por lo tanto, la lista de las fuentes bibliográficas utilizadas se confecciona automáticamente, a partir de las citas hechas en el texto.  Solo las fuentes citadas aparecerán en la bibliografía.

\begin{equation}
u(x) = 
  \begin{cases} 
   \exp{x} & \text{si } x \geq 0 \\
   1       & \text{si } x < 0
  \end{cases}
\end{equation}

Como se indicó anteriormente, se emplean \texttt{cite} y \texttt{citep} para hacer las citas.  Cual de estos dos comandos conviene utilizar, dependerá del contexto en que se haga la cita.  Según la redacción del párrafo, puede convenir que la fuente se indique en el formato ``Autor (año)'', pero en otros casos pudiera ser preferible que esta aparezca en el formato ``(Autor, año)''.

\begin{equation}
r(t) = \Re \left\lbrace \frac{\lambda}{4\pi} \left[ \frac{\sqrt{G_0} u(t) e^{-j2\pi r_0/\lambda}}{r_0} + \frac{\Gamma \sqrt{G_1} u(t-\tau) e^{-j2\pi r_1/\lambda}}{r_1} \right] e^{j2\pi f_c t}  \right\rbrace
\end{equation}
\input{contenido/4_sobre_LaTeX.tex}
\input{contenido/5_conclusiones.tex}

% 8. APÉNDICES
\appendix
\input{apendices/A_ejemplo}
\chapter{Un apéndice ejemplo con archivo anexo}

Para insertar archivos \texttt{.pdf} se utiliza el paquete \texttt{pdfpages}. Este paquete tiene opciones disponibles para la configuración del documento inserto.

Aquí hay dos ejemplos sencillos.

%\includepdf[pages=-,nup=2x2]{apendices/SIsummary.pdf}

\includepdf[pages=1]{apendices/LF353.pdf}

\includepdf[pages=-,scale=0.6,pagecommand={}]{apendices/Resumen.pdf}

\backmatter

% 9. BILIOGRAFÍA
\bibliographystyle{plain}
\bibliography{bibliografia/bibliografia.bib}

%%%%%%%%%%%%%%%%%%%
\end{document}
%%%%%%%%%%%%%%%%%%%